%!TEX root = ../main.tex

\section{Цель}
\spacing{1.5}

Сделать это необходимо на базе класса vector, созданном в первой
лабораторной (кто на три претендует, может воспользоваться стандартным
плюсовым классом вектора).

Соответственно, на входе в программу массив должен
быть пустым. Должна быть возможность с ним взаимодействовать
(добавлять/удалять элементы, просматривать массив и его размерность)
и, соответственно, сортировать указанным методом
