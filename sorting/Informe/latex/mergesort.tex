%!TEX root = ../main.tex

\section{Merge Sort}
\spacing{1.5}

Merge sort- Это алгоритм со сложностью \(\mathcal{O}(n\log{}n)\) где,
огласно с GeeksforGeeks это \cite{geek} "алгоритм « разделяй и властвуй». Он
делит входной массив на две половины, вызывает себя для двух половин,
а затем объединяет две отсортированные половины."

Алгоритм состоит из двух компонентов, первый - это функция merge(),
которая объединяет два массива в один отсортированный массив. Другой компонент
- mergesort(), который рекурсивно разбивает исходный массив на несколько
меньших и меньших массивов, затем объединяет эти меньшие неупорядоченные
массивы в больший упорядоченный массив и так далее, чтобы упорядочить
весь массив.


\subsection{Code MergeSort}
\singlespace

\begin{enumerate}
  \item \textit{mergesort.py}

  \lstinputlisting[language=Python]{code/mergesort.py}

\end{enumerate}
